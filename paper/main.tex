% main.tex
\documentclass[11pt]{article}
\usepackage{amsmath,amssymb,amsthm}
\usepackage{geometry}
\geometry{margin=2.5cm}
\usepackage{hyperref}
\title{Noncircular Proof of the Riemann Hypothesis via Sfinite Adelic Spectral Systems}
\author{Motanova et al.}
\date{October 2025}

\begin{document}
\maketitle
\begin{abstract}
We present a noncircular proof of the Riemann Hypothesis (RH) using Sfinite adelic spectral systems. Our construction avoids any direct input from the Riemann zeta function $\zeta(s)$, its Euler product, or its functional equation. Instead, we follow four independent pillars: geometry, symmetry, spectral uniqueness, and arithmetic emergence. The approach is extended to elliptic curves and BSD in the second part.
\end{abstract}

% --- Secciones principales ---
% 01_introduction.tex
\section{Introduction}
The Riemann Hypothesis (RH) asserts that all nontrivial zeros of the Riemann zeta function $\zeta(s)$ lie on the critical line $\Re(s) = 1/2$. It is one of the most fundamental open problems in mathematics, with deep connections to number theory, spectral theory, and quantum chaos. Despite partial advances by de Branges, Connes, Borcherds and others, no unconditional and noncircular proof has been achieved. The main obstacle is circularity: most spectral constructions presuppose the arithmetic structure they aim to derive. Our approach breaks this circle by grounding the proof in four independent pillars: geometry, symmetry, spectral uniqueness, and arithmetic emergence.
% 02_geometry.tex
\section{Geometry First}
We construct the universal multiplicative flow on $\mathbb{R}_{>0}$ and its generator $A_0 = 1/2 + iZ$, whose smoothed resolvent produces a Gaussian kernel in logarithmic variables. This step involves no arithmetic input and yields an entire function of exact order 1.
% 03_spectral_primes.tex
\section{Spectral Emergence of Primes}
The global scaling flow $S_u : x \mapsto e^{-u}x$ acts unitarily on $L^2(\mathbb{A}^\times/\mathbb{Q}^\times)$. For each place $v$, the local scaling operator $U_v$ induced by $x \mapsto \varpi_v^{-1}x$ satisfies the commutation relation $[U_v, S_u] = 0$ for all $u \in \mathbb{R}$. On the logarithmic coordinate $\tau = \log|x|_\mathbb{A}$, this action becomes $(U_v\varphi)(\tau) = \varphi(\tau + \log q_v)$, where $q_v$ is the local norm of the uniformizer. The primitive closed orbits have lengths $\ell_v = \log q_v$. These lengths emerge from Haar invariance and global compatibility, not from arithmetic assumptions.
% 04_functional_equation.tex
\section{Geometric Functional Equation}
The functional equation is forced not by Euler products, but by geometry. Let $J$ be the Mellin-Fourier involution $(Jf)(x) = x^{-1/2}f(x^{-1})$, $J^2 = \mathrm{Id}$. This operator implements Poisson-Radon duality on the adelic phase space, which is self-dual with respect to the symplectic form $\omega(x,\xi) = x\xi$. Acting on the generator $A_0 = 1/2 + iZ$, one finds $JA_0J^{-1} = 1 - A_0$. Consequently, the resolvent kernel and its Fredholm determinant obey the symmetry $D(1-s) = D(s)$.
% 05_positivity_debranges.tex
\section{DOI Positivity and de Branges Axioms}
Define the Hilbert space $H = \{f \in L^2(\mathbb{R}, e^{-\pi t^2}dt) : \mathrm{supp}\,\hat{f} \subset [0,\infty)\}$. On this space, the DOI-regularized operator $K_\delta$ acts by convolution with the Gaussian kernel $K_h$, which is positive-definite. By de Branges theory, a Hilbert space of entire functions satisfying axioms (H1)-(H3) forces all zeros of its canonical determinant to lie on the symmetry axis $\Re(s) = 1/2$.
% 06_paley_wiener.tex
\section{Spectral Uniqueness: Paley–Wiener with Multiplicities}
The positivity argument shows that all zeros are on the critical line. To identify $D(s)$ completely, including multiplicities, we appeal to determinacy theorems of Paley–Wiener type. The DOI-trace pairings of the explicit formula on two vertical lines $\Re(s) = \sigma_0$ and $\Re(s) = 1-\sigma_0$ determine the full zero divisor, including multiplicities. With Hadamard–Cartwright growth bounds and normalization at $+\infty$, this forces $D(s) \equiv \Xi(s)$.
% 07_zero_divisor.tex
\section{Noncircular Recovery of the Zero Divisor}
We did not assume $\zeta(s)$, its Euler product, or its functional equation. Instead: (1) Geometry provided the operator $A_0$ and the Gaussian kernel $K_h$; (2) Mellin–Fourier duality enforced the functional equation $D(1-s) = D(s)$; (3) DOI positivity and de Branges axioms forced all zeros onto $\Re(s) = 1/2$; (4) Paley–Wiener determinacy on two lines fixed the zero divisor uniquely. The canonical determinant $D(s)$ coincides with the completed Riemann function $\Xi(s)$, with identical zeros and multiplicities.
% 08_spectral_inversion.tex
\section{Spectral Inversion: Primes from Zeros}
Having identified $D(s) \equiv \Xi(s)$, we now recover the arithmetic content. The explicit formula, normally derived from Euler products, can be inverted from the zero set of $D$ alone. The distribution of prime powers $\Pi$ is uniquely reconstructed from the zero set $\{\rho\}$ of $D(s)$. No Euler product or global zeta input is required: the arithmetic structure emerges as the unique inversion compatible with the spectral geometry.

% --- Apéndices y referencias ---
% 09_appendix.tex
\section{Appendix}
Additional technical details, proofs, and numerical validations are provided here. See also the operator module and test suite for reproducibility.
% 10_bibliography.tex
\section{References}
\begin{thebibliography}{99}
\bibitem{Tate} J. Tate, Thesis, Princeton, 1950.
\bibitem{deBranges} L. de Branges, Hilbert Spaces of Entire Functions, Prentice-Hall, 1968.
\bibitem{Connes} A. Connes, Trace formula in noncommutative geometry, 1999.
\bibitem{Borcherds} R. Borcherds, Automorphic forms and Lie algebras, 1995.
\end{thebibliography}

\end{document}
