% 01_introduction.tex
\section{Introduction}
The Riemann Hypothesis (RH) asserts that all nontrivial zeros of the Riemann zeta function $\zeta(s)$ lie on the critical line $\Re(s) = 1/2$. It is one of the most fundamental open problems in mathematics, with deep connections to number theory, spectral theory, and quantum chaos. Despite partial advances by de Branges, Connes, Borcherds and others, no unconditional and noncircular proof has been achieved. The main obstacle is circularity: most spectral constructions presuppose the arithmetic structure they aim to derive. Our approach breaks this circle by grounding the proof in four independent pillars: geometry, symmetry, spectral uniqueness, and arithmetic emergence.